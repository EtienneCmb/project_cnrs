\documentclass[a4paper,11pt]{article}

\usepackage{style}

\begin{document}
\newtitle{Understanding the role of interactions in the human brain using deep recordings}{Research project}


%%%%%%%%%%%%%%%%%%%%%%%%%%%%%%%%%%%%%%%%%%%%%%%%%%%%%%%%%%%%%%%%%%%%%%%%%
%                         GLOBAL INTRODUCTION
%%%%%%%%%%%%%%%%%%%%%%%%%%%%%%%%%%%%%%%%%%%%%%%%%%%%%%%%%%%%%%%%%%%%%%%%%
\section{Context and project motivations}

At the microscale, the functional unit switched from the information-limited single neuron to population coding, where task-variables could be decoded from the coordinated activity of neural assemblies *(Yuste, 2015)*. Distributing the information has important implications, such as improving cognitive flexibility *(Fusi et al., 2016)*. A similar switch occurred at the macro scale where cognitive functions are supposed to be a network emerging property, i.e. the function produced by a set of regions can not be predicted from the role of individual regions. *(Varela, Siegel, Bressler etc.)*. In this conception, the role of a brain region depends on the network it is included in, and dynamic functional networks becomes the macro scale functional unit. 

Several lines of evidence are suggesting that, on one side, brain networks transiently organize into segregated subnetwork, performing specialized and similar computations. On the other side, the information processed within those segregated subnetworks is then thought to be integrated by multimodal networks *(Wolff et al., 2022)*. In parallel, the literature on the intrinsic timescales estimated from resting-state data proposes a hierarchy of the cortical regions based on their dominant oscillations, with unimodal sensory regions showing mainly short timescales (i.e. fast activity) and multimodal integrating regions like in the PFC showing long timescales (i.e. slow oscillations). Those observations can be linked to this general idea that local and segregated computations are mainly supported by fast activity (e.g. broadband and narrowband gamma) while long-range communication and integration should be supported by synchronization between slow oscillations. However, it is still unclear how to coordinate distant and segregated fast activity. Cross-frequency coupling has been suggested as a plausible mechanism for coordinating distant fast activity by means of the influence of slow oscillations *(Canolty, Bassett etc.)*.

\paragraph{Understanding cognitive functions :} okok

\paragraph{The emergence of network neuroscience :} ook

\begin{oquest}
    \textbf{Open-questions}
    \begin{enumerate}
        \item How interactions between brain regions are modulated during cognitive tasks?
        \item How the structural connectivity shapes interactions?
    \end{enumerate}
\end{oquest}

%%%%%%%%%%%%%%%%%%%%%%%%%%%%%%%%%%%%%%%%%%%%%%%%%%%%%%%%%%%%%%%%%%%%%%%%%
%                        PROJECT INTRODUCTION
%%%%%%%%%%%%%%%%%%%%%%%%%%%%%%%%%%%%%%%%%%%%%%%%%%%%%%%%%%%%%%%%%%%%%%%%%
\section{Introduction to the research project}

\begin{bigpic}
    My research project consist in investigating how brain regions interact
    
    \begin{description}
        \item[{\large Axis 1} :]
        \item[{\large Axis 2} :]
    \end{description}
    
\end{bigpic}


%%%%%%%%%%%%%%%%%%%%%%%%%%%%%%%%%%%%%%%%%%%%%%%%%%%%%%%%%%%%%%%%%%%%%%%%%
%                           AXIS 1 : NEURO
%%%%%%%%%%%%%%%%%%%%%%%%%%%%%%%%%%%%%%%%%%%%%%%%%%%%%%%%%%%%%%%%%%%%%%%%%

\section{Axis 1 : Origins and role of brain interactions}

\subsection{Linking structural and functional connectivity}

\subsection{Modulation of interactions during cognitive tasks}

\subsection{Interactions during resting state}

%%%%%%%%%%%%%%%%%%%%%%%%%%%%%%%%%%%%%%%%%%%%%%%%%%%%%%%%%%%%%%%%%%%%%%%%%
%                    AXIS 2 : METHODOLOGICAL
%%%%%%%%%%%%%%%%%%%%%%%%%%%%%%%%%%%%%%%%%%%%%%%%%%%%%%%%%%%%%%%%%%%%%%%%%

\section{Axis 2 : }

\subsection{Identification of cognitive brain networks}

\citep{combrisson2022grouplevel}

\subsection{Contribution to the Open-science}

\citep{combrisson2017sleep,combrisson2019visbrain,combrisson2020tensorpac}










\phantomsection
\addcontentsline{toc}{section}{References}
\bibliography{refs}











\newpage
\section{Ideas}

\subsection{Bricks to connect}

\begin{description}
    \item[Connectivity :] Functional, Structural, Resting state
    \item[Higher order interactions :] static, dynamic, computational limitations, plotting, summarizing
    \item[Mixed selectivity and degeneracy :] PFC, cognitive flexibility, how it shapes interactions
    \item[Learning :] CausaL and PBLT
    \item[Task :] New task? localizer
    \item[Data :] Human sEEG, MEG?
\end{description}

\subsection{Global ideas}

\begin{description}
    \item[Functional connectivity on sEEG data] several advantages of sEEG :
    \begin{enumerate}
        \item Make the bridge between M/EEG non-invasive results and LFP in animals
        \item Resting-state FC (\textbf{collaboration with Karim} - DMN) and FC during cognitive tasks
        \item See \cite{helfrich2019cognitive} for all the arguments and advantageous of using sEEG data
        \item Correlation between structural connectivity and functional connectivity (Mélina - diffusion MRI - Pragues)
        \item Generalization to healthy subjects : we have the MEG and sEEG CausaL data. We could analyze both and see if we have similar results and answer to main criticism about the generalization to healthy subjects
    \end{enumerate}
    \item[Disentangling mixed selectivity using the Functional Connectivity] Brain regions are involved in many aspects and are encoding many external features. Neurons are encoding components of items (mixed selectivity) and task-related features can be retrieved using the information from population of neurons (population coding). The same principle can be applied using the FC. Some brain regions might not be specific (e.g. OFC, insula) but the link between them might be.
    \item[Influence exerted by a brain region on another] since task-relevant features are encoded in several brain areas, future research should investigate how the encoding in a brain region modify the encoding in a different brain region
    \item[Synchronization of neuronal populations] Synchronization of neuronal populations enhance the power correlation \citep{womelsdorf2007modulation}. Local power is modulated according to behavioral variable. \textbf{What is the link between phase-synchronization and behavioral variables?}
    \item[Features decomposition] External features have been suggested to be decomposed into small components \citep{o2021hierarchical}. \textbf{Does it make sense to study the PFC in isolation of the rest of the brain?}
    \item[Cross-frequency communication] How can distant brain regions communicate considering that the local activity contains peaks in different frequencies? n:m connectivity
    \item[FC vs. Multi-variate statistics] It's possible to use feature-selection through ML algorithms to identify and associate "top brain regions"for decoding a feature. This approach can be used to minimize the redundancy carried by each region. However, this don't necessary implies that the brain regions are synchronized.
    \item[PID `I\&' HOI vs. Multi-feature selection] Multi-features selection basically is looking for brain areas association that increase the decoding. Hence, this is probably similar to high-order synergistic effects. We could investigate this question.
\end{description}


\subsection{Important papers}

\begin{enumerate}
    \item Laurence Hunt \cite{hunt2017distributed}
\end{enumerate}

\section{Project}

\subsection{Information theory vs. Machine learning}

\begin{bigpic}
    \begin{center}
        \textbf{Investigate feature selection for cognitively relevant regions} 
    \end{center}
\end{bigpic}


\end{document}